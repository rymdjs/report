Detta projekt syftar till ta fram ett modulärt bibliotek i JavaScript som är avsett för utvecklare och går under namnet \emph{Rymd}. Biblioteket tillhandahåller säker filöverföring direkt mellan webbläsare. Projektet innefattar utvärdering av moderna webbtekniker för att säkerställa de mest lämpade alternativen.

Drivkraften till projektet kommer från viljan att öka enskilda användares kontroll över sin egen data. Detta i kontrast till dagens trender där externa parter driver centraliserade tjänster för exempelvis chattapplikationer, datalagring och filöverföring. Något som genomsyrar hela projektimplementationen är säkerhet, decentralisering, anonymitet samt modularitet. 

Som demonstration av Rymds funktionalitet skapades även en webbapplikation vid namn \emph{Shuttle}, där givna implementationer för kärnfunktionaliteten i Rymd är givna. För datalagring används IndexedDB medan peer-to-peer-kommunikationen utnyttjar WebRTC. Vidare används det ej färdigställda Web Cryptography API för kryptografiska operationer såsom kryptering, dekryptering, signering, med mera. För att lagra kryptografiska nycklar används kryptovalutan Namecoins så kallade \emph{blockchain} där publika nycklar som används för verifiering av identiteter kan hämtas ut med hjälp av användaralias.

Projektet mynnade i slutändan ut i en fungerande fildelningsplattform, med vissa brister i säkerheten. Detta härleds direkt till det tidiga utvecklingsstadiet i de webbteknologier som används. Då webben utvecklas i en rasande takt av både företagen bakom webbläsare samt av standardiseringsorgan är vi däremot säkra att detta rättas till så småningom. Med säker kommunikation direkt mellan klienter kommer anonymitet få ett stort fäste. Vi ser med spänning på vad framtiden har att erbjuda.
