Detta projekt syftar till att undersöka moderna webbteknikers potential att möjliggöra säker filöverföring direkt mellan webbläsare. Något som genomsyrar hela projektimplementationen är säkerhet och decentralisering. Det mesta av problemformuleringarna krestade kring detta i samband med kapaciteten hos teknologier som körs direkt i webbläsaren – något som traditionsenligt inte varit speciellt säkert samt har berott på centrala servrar.

Till de resulterande produkterna hör huvudsakligen ett modulärt JavaScriptbibliotek som ger utvecklare möjlighet att bygga egna slutprodukter som injicerar egna alternativ till kärnfunktionaliteten. Som försäkran att konceptet fungerar enligt kraven utvecklades även en sådan slutprodukt.

Implementationerna använder sig mestadels av nyutvecklade externa bibliotek och tekniker, vissa så nya att en fullständig specifikation ännu ej är fastslagen. Ett resultat av detta är att de har dålig täckning i webbläsare på marknaden. Ingen webbläsare tillhandahåller i nuläget alla teknikerna fullt ut. Google Chrome har däremot tillräckligt stöd för att en körbar prototyp skulle kunna tas fram (fast då med framförallt brister i säkerheten).

Webben utvecklas i rasande takt och kommer inom snar framtid att möjliggöra projekt med liknande funktionalitet fullt ut. Detta öppnar dörrar till tjänster med många användningsområden. Med säker kommunikation direkt mellan klienter kommer annonymitet få ett stort fäste. Vi ser med spänning på vad framtiden har att erbjuda.


%För beständig datalagring i webbläsaren utsågs IndexedDb som bästa alternativ medan kommunikationen tillhandahålls av peer.js med WebRTC som grund.