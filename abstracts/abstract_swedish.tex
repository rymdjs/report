Detta projekt syftar till ta fram ett modulärt bibliotek i JavaScript som är avsett för utvecklare och går under namnet \emph{Rymd}. Biblioteket tillhandahåller säker filöverföring direkt mellan webbläsare. Projektet innefattar granskning och analys av moderna webbtekniker för att säkerställa de mest lämpade alternativen.

Något som genomsyrar hela projektimplementationen är säkerhet, decentralisering, annonymitet samt modularitet. Vidare kommer drivkraften till projektet från viljan att ta bort beroendet och överlämningen av kontroll som uppkommer i och med centraliserade tjänster såsom chattapplikationer, datalagring och filöverföring.

Som försäkran att biblioteket fungerar enligt kraven skapades även en slutprodukt som utnyttjar modulariteten i systemet genom att tillhandahålla kärnfunktionaliteternas implementationer till Rymd genom moduler. För lagring av data som består mellan sessioner används IndexedDB medan kommunikationen utnyttjar WebRTC. Vidare är används (det ej färdigställda) biblioteket Web Cryptography API till kryptografiska operationer såsom kryptering, dekryptering, signering, med mera. För att lagra kryptografiska nycklar används kryptovalutan Namecoins så kallade \emph{blockchain} där de är mappade mot användaralias.

Projektet mynnade i slutändan ut i en fungerande fildelningsplattform, men det finns framförallt brister i säkerheten. Detta härleds direkt till mognaden i de webbteknologier som används. Då webben utvecklas i en rasande takt är vi däremot säkra att detta kommer att rättas till så småningom. Med säker kommunikation direkt mellan klienter kommer annonymitet få ett stort fäste. Vi ser med spänning på vad framtiden har att erbjuda.
