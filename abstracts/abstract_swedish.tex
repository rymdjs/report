Detta projekt syftar till ta fram ett modulärt bibliotek i JavaScript som är avsett för utvecklare och går under namnet \emph{Rymd}. Biblioteket tillhandahåller säker filöverföring direkt mellan webbläsare. Projektet innefattar utvärdering av moderna webbtekniker för att säkerställa de mest lämpade alternativen.

Huvudproblematiken kretsar kring hur systemet kan göras säkert, decentraliserat och modulärt med hjälp av moderna webbteknologier. Ett sådant system kan användas för att öka användares kontroll över traditionellt centraliserade tjänster, såsom chatt och filsynkning.

Som demonstration av Rymds funktionalitet skapades även en webbapplikation vid namn \emph{Shuttle}, där givna implementationer för kärnfunktionaliteten i Rymd är givna. För datalagring används IndexedDB medan peer-to-peer-kommunikationen utnyttjar WebRTC. Vidare används det ej färdigställda Web Cryptography API för kryptografiska operationer såsom kryptering, dekryptering, och signering. För att lagra kryptografiska nycklar används kryptovalutan Namecoins så kallade \emph{blockchain} där publika nycklar som används för verifiering av identiteter kan hämtas ut med hjälp av användaralias.

Projektet mynnade i slutändan ut i en fungerande fildelningsplattform, med vissa brister i säkerheten. Dessa brister kan direkt härledas till det tidiga utvecklingsstadiet i de webbteknologier som används. Då webben utvecklas i en rasande takt av både webbläsare och standardiseringsorgan är vi dock säkra på att detta rättas till så småningom.
