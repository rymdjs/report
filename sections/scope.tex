\section{Scope}

The project is planned to encompass two end products: \emph{Rymd} and \emph{Shuttle}.

\begin{description}
\item[Rymd] is the main outcome and end goal of the project. It should solve authentication and data transfer between nodes, while also providing encryption and storage of resources locally. It should be usable as a drop-in module by any web client side code, such as a regular front-end web application, browser extension, or widget.

\item[Shuttle] is a proof-of-concept prototype using Rymd to show its functionality. It can be seen as an executable evaluation of Rymd and will be briefly discussed in this report. Shuttle should be a working example of a file-sharing application that leverages Rymd to provide all these features.
\end{description}

% TODO Förklaras blockchain?
The system will not deal with version management, syncing, merging resources, and history. Neither will the upload of the users’ keys (used for encrypting data) to the Namecoin blockchain be solved by Rymd, since it involves payment logistics – a subsystem not in our scope.

Leaking of certain kinds of metadata will not be addressed. This includes information on who is communicating with who, since this is a very difficult issue far beyond the scope of this project. Also, network operators will likely be able to make a rough estimate of resource size based on the amount of data transferred, since this is considered a reasonable privacy-performance tradeoff as long as transfers are padded enough so that the estimation can only ever be so vague.

Some of the technologies involved in this project have been developed quite recently, which means that even some of even the latest browsers might lack support. The final product will most likely not work on all types of devices and browsers.
