\lettrine[lines=2, findent=2pt]{I}{n a day} and age where people all over the world own more than one digital device\cite{OFCOMa:Online}\cite{OFCOMb:Online}, there is a growing need for services which let users easily store, access, synchronize, and share personal files. The internet makes it possible to store data \emph{in the cloud} and access it from a web browser or a designated client. Instead of storing files on physical media, it is today a common practice to use services such as Dropbox, Google Drive, or Apple's iCloud for home and business matters. In November 2013, Dropbox reached 200 million users\cite{Constine:2013:Online}. Google Drive is integrated across a large number of Google's products, as is the case with Apple's iCloud, which stores and synchronizes personal preferences across devices and applications\cite{CloudTrend:Online}.

The existing mainstream services mentioned above are centralized, which means that the files and resources stored with them are placed on central servers somewhere on the internet. This report describes the results of developing a file sharing system that runs independently on each participant's computer and doesn't rely on any central server - a decentralized system. Focus will be on security and open web standards.


\section{Background}
Most of the existing technologies and protocols that constitute the internet, as well as the services running on it, are by design decentralized and promote the design of distributed systemsInternet \cite{InternetDecenterlized:Online}. On the application layer, transfers of the hosted large files in peer-to-peer scenarios are to a growing extent conducted in a distributed fashion using the Bittorrent protocol, to the point where it is becoming the de facto approach for many use cases and areas. In others, the transition to distributed transfers and storage of data is still in its infancy. In technology and developer communities, \emph{distributed} and \emph{decentralized} have become buzzwords. The norm today is to use distributed approaches for things such as source version control, data storage, heavy computations and content delivery.
However, both users, developers and businesses are moving more and more data to an arbitrary \emph{cloud} on the internet. Companies such as Google and Dropbox provide servers for storage of data: A practice that poses several security concerns. Recent news on government infiltration of these services, as well as the revelation of Microsoft  defense of private investigations in users' Hotmail inboxes, raises the issue of centralized storage beyond users' control. The internet itself has always been decentralized, and resembles a conventional graph structure with nodes and paths. By centralizing information and giving up a \emph{thin server-fat client} concept, one deviates from the fundamental idea of a decentralized network.

There is currently a clear transition of user-space applications and services from native binaries to web applications running inside a web browser. Recent initiatives such as Google Chrome Apps, Adobe Phonegap and Mozilla Firefox OS are starting to bridge the gap between \emph{web apps} and \emph{native apps} even more, both for mobile and desktop environments. These applications are implemented using what is casually referred to as \emph{HTML5} or, more accurately, the open web stack – an umbrella term for technologies such as HTML, CSS and JavaScript, which are based on open standards. Web applications are becoming ever-increasingly powerful in areas of software engineering and computer science, even though many standards are still in their infancy. Browser implementation and support is still unstable in many areas, leaving much to cover. Even so, technologies for functionality that was earlier exclusively for native applications are now available for any developer to use in modern, cutting edge web browsers. Notable examples are peer-to-peer video chat, local file storage, powerful encryption methods, and real time full-duplex communication.

Following these trends, a natural consequence is a peer-to-peer distributed data-syncing protocol implemented purely on the open web stack utilizing cryptographic keys for access control. This would hopefully act as a stepping stone facilitating the development of user-friendly and convenient, yet secure and privacy-protecting distributed implementations of services such as personal file syncing, media sharing and private communication.

\section{Purpose}

This project aims to build a distributed peer-to-peer encrypted system for storage and communication of resources packaged in a library, with a web browser as the only client-side dependency. In this report, the results are presented, along with the state of modern web standards as these technologies are researched analyzed in terms of what they provide to build such as system.

\begin{description}
\item[Rymd] is the main outcome and end goal of the project. It will solve authentication and data transfer between nodes, as well as providing encryption and storage of resources locally. It should be usable as a drop-in module by any web client side code, such as a regular front-end web application, browser extension, or widget.
\item[Shuttle] is a proof-of-concept prototype using Rymd to show its functionality is also briefly discussed in this report.
\end{description}

Below are the important goals and requirements of Rymd:

\begin{description}
  \item[Privacy]. Only users that are given explicit access to a resource should be able to deduce anything useful about its content. No central entity, such as a server administrator or network operator, should be able to extract incriminating information about a client. Users should be able to trust that they know who they are communicating with. No network operator or server administrator should be able to forge identities in a way that can not be detected by a user.

\item[Security]. Encryption in all layers, from resource storage to data transfer.

\item[Reliability]. If any server goes down and can not be recovered, no damage should be done to the network as a whole as long as anyone can host a new server using the same source code.

\item[Modularization and agnosticism]. The system as a whole should not depend on particular implementations for resource storage, key storage or which protocol to use for data transfer. If a developer wants to, they should be able to easily plug in their own alternative implementation module.

\end{description}

Shuttle should be a working example of a file-sharing application that leverages Rymd to provide all these features.

\section{Problem}
\label{sec:problem}

% TODO Robert: Make shorter and more concrete?
There are several sub-problems in a system of this type that the project needs to address:

\begin{itemize}
\item Decentralization of system logic
\item Peer identity verification
\item Resource storage
\item Resource identification
\item Communication flow and transfer initiation
\end{itemize}

These points affect the architecture as presented in the following sections.

\subsection{Decentralization of system logic}
In a truly distributed system, it is necessary to avoid having crucial system logic and data on a central server. The functionality of the system should not rely on the availability of any specific server. If servers are needed for any reason, they should not store persistant or sensitive data and be easily replacable with new servers running the same software. Temporary downtime can be accepted in this case. Since clients do not know what software their peers are running, all information from them must be considered untrusted until verified.

\subsection{Peer identity verification}
Each user of the system will be associated with a self-generated private-public pair of asymmetric cryptographic keys. With knowledge of the public keys of their peers, there are standardized identity verification protocols used on a session-to-session basis. Regardless of the authentication protocol used, there is always a chicken-and-egg problem with the distribution of public keys and how to tie them to identities. In order to trust the validity of the key provided from another entity, the user puts trust in that entity. Traditionally, there are two types of Public Key Infrastructures (PKIs) with different ways to address this:

\begin{itemize}
  \item A Web of Trust, as often utilized in OpenPGP \cite{Maurer:1996}. Here, a user has a list of peers that they trust - trusted introducers. If they receive a public key and associated identity signed by one of their trusted introducers, they will know that the trusted introducer has verified the connection between the identity and the public key. In this way, an active user will steadily grow their network of trusted introducers. One needs to have a network of dependable and active peers in order to successfully participate in a Web of Trust.

\item A PKI centered around one or several Certificate Authorities (CAs). Here, there is a predefined list of authorities that are trusted to sign participants public keys. This creates a centralized network and puts a lot of trust in the CAs. SSL utilizes this approach and there are several historical examples of when this trust has been broken (more recently in the Diginotar hack of 2011).
\end{itemize}

\subsection{Resource storage}
Usability, security and adherence to public web standards are three highly prioritized properties that make the question of how to locally store resources on clients a difficult one. The FileSystem API\footnote{http://w3c.github.io/filesystem-api/Overview.html} enables access of the local filesystem. Implementation has started in some browsers, but the standard is now considered dead\cite{W3C:2014}. Local file access could be very useful – but without cross-platform support, it are considered out of the question. A secure way to store the encryption keys for encrypted resources also needs to be determined.

\subsection{Resource identification}
It is desirable for resources to have identifiers that are memorable, secure, and unique. Resource checksums will have to be communicated and verified by peers before accepting a transfer of resource data.

\subsection{Communication flow and transfer initiation}
In order for nodes to be able to share data, they need a way to connect to each other. They also need to do this in a secure manner in order to prevent potential vicious third parties listening on a connection from making any sense of retrieved data. In other words critical parts should not be sent in raw form, but rather be encrypted. When considering security aspects there are essentially three questions that need to be answered regarding the issue of connecting nodes:

\begin{itemize}
\item How can a node find another node to begin with (peer discovery)?
\item When a node has been found, how can a connection be established?
\item What data needs to be encrypted in order to ensure the integrity of the system?
\end{itemize}

Answering these questions and finding the corresponding best technology was the focus of this research area. In recent years there has been a trend driving more and more of tools and services on the web towards user collaboration. A natural step in this trend is initiatives such as WebRTC and CU-RT-Web which enable direct peer-to-peer communication.

The project will be scoped to encompass two parts:
\begin{itemize}
\item A library for dealing with encryption, storage and data transfer communication between clients.
\item A proof-of-concept prototype implementation (like a “front-end Dropbox” app) that runs in a modern web browser.
\end{itemize}
The system will not deal with version management, syncing, merging resources, and history. Neither will the upload of the users’ public encryption keys to the block chain be solved by Rymd, since it involves payment logistics – a subsystem not in our scope.

Leaking of certain kinds of metadata will not be addressed. Namely, information on who is communicating with whom, since this is a very difficult issue far beyond the scope of this project. Also, network operators will likely be able to make a rough estimate on resource size based on the amount of data transferred, since this is considered a reasonable privacy-performance tradeoff as long as transfers are padded enough that the estimation can only ever be so vague.

Some of the technologies involved in this project are quite recently developed, which means that some of even the latest browsers might lack support for some of them. The final product will most likely not work on all types of devices and browsers.



\section{Structure}
  \label{sec:structure}

Chapter~\ref{chap:methodology} describes the different parts of this project methodology-wise. The main part of this report can be viewed as a three-step process: a theoretical background; a design, evaluation and analysis chapter; and an implementation chapter.

In chapter~\ref{chap:tech_background}, fundamental information about relevant theories and technologies is given in order to supply the reader an understanding of the field. Chapter~\ref{chap:similar} surveys the current landscape and positions this project among other related work. An analysis and overall system design is described in chapter~\ref{chap:design}, which concludes in a specification of the underlying modules. The low-level implementations of these are described in chapter~\ref{chap:implementation}.

The final part of this report involves discussion and conclusion in chapters~\ref{chap:discussion}~and~\ref{chap:conclusion}, where the results of the project are presented and discussed.

