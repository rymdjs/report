\section{Similar technologies}
There is a plethora of disparate technologies for distributing and syncing files between peers that at first glance may look very similar to Rymd. Below are some of the more well-known and similar pieces of software and point out similarities and differences.
\begin{description}
  \item[Bittorrent Sync] (Bittorrent Inc, 2012) is a distributed peer-to-peer multi-way file syncing software using the Bittorrent protocol for file transfers. Synced folders are mapped directly to the underlying file system, and each folder is encrypted using a shared secret key. Public-key cryptography is not employed, and there are only closed-source binary clients using Bittorrent, Incs network available. While they do have a developer API, it requires developer keys issued from Bittorrent Inc.
  \item[RetroShare] markets itself as a Friend-2-Friend decentralized communication platform. It uses GPG to create a Web of Trust between peers. It is, however, a very large project: The application provides file-sharing, instant messaging, discussion forums, e-mail, VoIP and group chat. It is open source and distributed as cross-platform binaries.
  \item[ShareFest] is a peer-to-peer one-to-many file-sharing web based software using WebRTC data channels. ShareFest can be seen as a more limited and primitive version if what Rymd aims to be: ShareFest can share files over WebRTC channels, but does not accommodate for authentication, persistence or local encryption. It does, however, operate on a mesh network similar to Bittorrent. Other similar WebRTC-based P2P file sharing web applications but without additional cryptographic properties include RTCCopy and ShareDrop.
  \item[Freenet] (2000) is one of the first "darknets", consisting of a distributed, decentralized data store that uploads files with strong anonymity across a network. Each node in the network also acts as a cache for the content stored in the network. Files are generally split up in parts that are distributed, and when fetching files it is unfeasible to determine the origin and sender of the files. Focusing on anonymity, free speech and plausible deniability, the encryption is done in the communication and storage layers. Because of this design, Freenet is quite slow. Files can be retrieved using the cryptographic key used to upload it. Freenet is free software built with Java.
  \item[Tahoe-LAFS] or Tahoe Least-Authority Filesystem (The Tahoe-LAFS Software Foundation, 2006) is a distributed, encrypted and redundant file system. It distributes encrypted files across a predetermined set of servers and allows sharing of both mutable and immutable files. There is a web-interface, but like all other user-interfaces it has to go through a "gateway" where encryption and server-communication is performed. Users will typically run their own gateways and will thus need to accommodate for hosting for them. 

\end{description}
