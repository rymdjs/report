\section{Background}
\lettrine[lines=3]{M}{ost of the} existing technologies and protocols that constitute the internet, as well as the services running on it, are by design decentralized and promote design of distributed systems. On the application layer, transfers of hosted larger files in peer-to-peer scenarios are to a growing extent conducted in a distributed fashion using the Bittorrent protocol, to the point where it is becoming the de facto approach for many use cases and areas. In others, the transition to distributed transfers and storage of data is still in its infancy. In tech and developer communities, "distributed" and "decentralized" have become buzzwords. The norm today is to use distributed approaches for things such as source version control, data storage, heavy computations and content delivery.
However, both users, developers and businesses are moving more and more data to an arbitrary "cloud" on the internet. Companies such as Google and Dropbox provide servers for storage of data related to their services, a practice that poses several security concerns. Recent news on government infiltration of these services, as well as the revelation of Microsoft conducting and even defending private investigations in users' Hotmail inboxes, raises the issue of centralized storage beyond users' control. The internet itself has always been decentralized, and resembles a conventional graph structure with nodes and paths. By centralizing information and giving up a "thin server-fat client" concept, one deviates from the fundamental idea of a decentralized network.

There is currently a clear transition of user-space applications and services from native binaries to web applications running inside a web browser. Recent initiatives such as Google Chrome Apps, Adobe Phonegap and Mozilla Firefox OS are starting to bridge the gap between "web apps" and "native apps" even more, both for mobile and desktop environments. These applications are implemented using what is casually referred to as "HTML 5" or, more accurately, the open web stack – umbrella terms for technologies such as HTML, CSS and JavaScript, which are based on open standards. Web applications are becoming ever-increasingly powerful in areas of software engineering and computer science, even though many standards are still in their infancy and browser implementation and support is still patchy in many areas. Technologies for building functionality that was earlier exclusively for native applications are now available for any developer to use in modern, cutting edge web browsers. Notable examples are peer-to-peer video chat, local file storage, black box encryption and real time full-duplex communication.

Following these trends, a natural conclusion is a peer-to-peer distributed data-syncing protocol implemented purely on the open web stack utilizing cryptographic keys for access control. This would hopefully act as a stepping stone facilitating the development of user-friendly and convenient, yet secure and privacy-protecting distributed implementations of services such as personal file syncing, media sharing and private communication.
