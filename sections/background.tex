\section{Background}

Most of the existing technologies and protocols that constitute the internet, as well as the services running on it, are by design decentralized and promote the design of distributed systems~\cite{InternetDecenterlized:Online}. On the application layer, the hosting of large files are to a growing extent conducted in a distributed peer-to-peer fashion using the Bittorrent protocol, to the point where it is becoming the de facto approach for many use cases and areas. In others, the transition to distributed transfers and storage of data is still in its infancy. In technology and developer communities, \emph{distributed} and \emph{decentralized} have become buzzwords. The norm today is to use distributed approaches for things such as source version control, data storage, heavy computations, and content delivery.

However, users, developers, and businesses alike are moving more and more data to an arbitrary \emph{cloud} on the internet. Companies such as Google and Dropbox provide servers for data storage: An approach that poses several security concerns. Recent news on government infiltration of these services, as well as the revelation of Microsoft defense of private investigations in a users' Hotmail inboxes~\cite{Frank:2014}, raises the issue of centralized storage beyond users' control. The internet itself has always been decentralized and resembles a conventional graph structure with nodes and paths. By centralizing information and giving up a \emph{thin server-fat client} concept, one deviates from the fundamental idea of a decentralized network.

Simultaneously, there is currently a clear transition of user-space applications and services from native binaries to web applications running inside a web browser. Recent initiatives such as Google Chrome Apps, Adobe PhoneGap, and Mozilla Firefox OS are starting to bridge the gap between \emph{web applications} and \emph{native applications} even more, both for mobile and desktop environments. These applications are implemented using what is casually referred to as \emph{HTML5} or, more accurately, the open web stack – an umbrella term for technologies such as HTML, CSS, and JavaScript, which are defined by open standards. Web applications are becoming increasingly powerful in areas of software engineering and computer science, even though many standards are still in their infancy. Browser implementation and support is still unstable in many areas, leaving much to cover. Even so, technologies for functionality that was earlier exclusively for native applications are now available for any developer to use in modern, cutting edge web browsers. Notable examples are peer-to-peer video chat, local file storage, powerful encryption methods, and real-time full-duplex communication.

Following these trends, a natural consequence is a peer-to-peer distributed data-syncing protocol implemented purely on the open web stack utilizing cryptographic keys for access control. This would hopefully act as a stepping stone facilitating the development of user-friendly and convenient, yet secure and privacy-protecting, distributed implementations of services such as personal file synchronization, media sharing, and private communication.
