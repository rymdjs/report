Standards and technologies for web applications are being rapidly developed, and the boundaries for what is possible to achieve in a web application is being continously pushed by browser vendors and standards groups. Of relevance to Rymd, the Web Real-Time Communications (WebRTC) protocol \cite{WebRTC:Online} has become usable for arbitrary data streams in major browsers in 2014. Also of relevance, the still unfinalized Web Crypto API \cite{WebCrypto:Online} is currently available in an experimental stage in recent months at the time of this writing. The goals of Rymd has thus become technically viable in a web environment as of very recently.

Furthermore, there is currently a lot of work underway in the field of distributed secret communication. Notable projects that share similar ideas or have inspired Rymd are mentioned under section~\ref{sec:similar}. Also worth noting is the new field of cryptocurrencies, which work by distributing a cryptographically based ledger over an entire network. The first and most well-known is Bitcoin \cite{Bitcoin:Online}, but there are also subsequent currencies that extend the original idea beyond that of a traditional currency to a system that can be used for a wide range of applications. The first worth noting is Namecoin \cite{Namecoin:Online}, which adds a global and secure key-value store. Another interesting initiative is Ethereum \cite{Ethereum:Online}, a cryptocurrency with contracts that not only allow storage of arbitrary data in the blockchain, but can also be scripted with a Turing complete programming language and can therefore be used to implement arbitrary systems. A system like Ethereum could be very interesting to explore for a project like Rymd, but it is still in such an early stage that it is deemed to unstable to be useful at this point. Namecoin is currently considered a good candidate for key distribution.

Keybase \cite{Keybase:Online} is another recent initiative that intends to solve the distribution of public keys. It is essentially a HTTP-interface that maps keys to identities. While commandable, it again raises the issue of centralized storage. Systems relying on Keybase put a lot of trust on the availability and integrity of their service.
