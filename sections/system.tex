This chapter describes the technical foundations of the system and how the problems described in section~\ref{sec:problem} have been addressed in Rymd.

% TODO: Why encrypt-then-MAC? Refer to http://cseweb.ucsd.edu/~mihir/papers/oem.html, explained here: http://crypto.stackexchange.com/questions/202/should-we-mac-then-encrypt-or-encrypt-then-mac
%TODO: Motivate choice of AES-CBC-256 (in implementation?)

\subsection{Resources}
Full access to a resource implies possession of three things: The encrypted resource data, they cryptographic key used to encrypt said data and the metadata describing the resource. The creation of a new resource goes like this:
\begin{enumerate}
  \item Metadata is generated. Metadata consists of resource name, author identity, MIME type, a randomly generated GUID, incrementing file version (always $1$ in the case of new resource) and a timestamp.
  \item A resource-specific symmetric cryptographic key is generated. In the default implementation, a 256 bit AES-CBC key is used.
  \item The resource data is encrypted using the resource key.
  \item A resource hash is calculated based on the metadata and encrypted data. 
  \item Add the hash to the metadata.
  \item Combine the metadata and the encrypted data to form the internal representation of the resource.
  \item Save the key in a local key store.
\end{enumerate}

To save a resource locally, a user-specific symmetric \emph{master key}, generated at the time of first access, is used to encrypt the metadata before saving it and the encrypted resource data to a local resource store.


\section{Peer identity verification}
\label{sec:authorization}
% Robert

For a truly decentralized system, it is not acceptable to adapt a CA-entered approach. While a Web of Trust is interesting, it might be too cumbersome for users. This issue is addressed in "Zooko's Triangle" (See figure ~\ref{fig:zooko}), stating that no system assigning names to participants in a network can have the property that names are secure, decentralized and meaningful at the same time. This conjecture has since been proven false by the design of systems such as the blockchain of the cryptocurrency Namecoin, which effectively acts as a cryptographically secured distrubuted hash table (DHT) with unique keys. Users can reserve a name and assign to it a value of their choice by the cost of a small amount of the Namecoin currency (at the time of this writing 0.01 NMC \cite{Namecoin:2014:Online}, which equals roughly 0.03 USD \cite{CryptoCoinCharts:2014:Online}).

Rymd therefore utilizes a DHT for storage of keys to achieve all of these goals: The distributed nature of cryptocurrencies makes it decentralized; peers can choose their own names (identities), giving meaningful names; the small monetary fee required to register a name makes it both secure and prevents massive name-squatting by malicious third parties.

\begin{figure}[h]
\centering
\includegraphics[width=\textwidth,height=0.2\paperheight,keepaspectratio
]{figures/Zooko_s_Triangle}
\caption{Zooko's Triangle, with the edges representing the achievable combinations of features \cite{Zooko:2001:Online}}
\label{fig:zooko}
\end{figure}

\section{Decentralization}
% Robert

Assuming that the system can utilize a DHT such as a cryptocurrency blockchain for storage of the public part of RSA key pairs, the issue of how to interface a web application with the blockchain in a way that allows for verification of identities without putting too much trust in the HTTP/cryptocurrency gateway also needs to be addressed. Additionally, as previously stated, the initial insertion of the key requires monetary resources, and is perhaps something that should be solved outside of Rymd.
While the public key can be stored in a DHT, private keys need to be stored securely on each client, preferably without giving client code any direct access to the raw keys. How the initial generation of keys are performed also needs consideration. Finally, a secure way to store the encryption keys for encrypted resources needs to be addressed. The question of how these resource-associated secret keys are distributed is deemed an implementation-specific question and will be out of scope for Rymd, but handled in Shuttle.

% "Centralized control – Distributed Data Architectures"
% http://highscalability.com/blog/2014/4/7/google-finds-centralized-control-distributed-data-architectu.html

\section{Modularity}

% TODO Robin

The system was designed from the ground up to use interchangeble parts. This was achieved by identifying key features and separating them into individual modules. These were then used and intertwined in a central hub.

Advantages

Dependency injection
