Due to the project group’s uncertainty of the state and capabilities of the technologies for the planned implementation, the project was split into two parts: a research phase and an implementation phase. This chapter intends to further detail the results of each phase and how it relates to the end product.

\section{Research}

Before actual work could commence, research had to be made about open web technologies. The usage of open web technologies was a requirement of the project’s end product, Rymd, which meant that no native code could be written as part of the system and that the quality of the product would be completely dependant on the state of existing APIs and tools of web development. Thus the research was made in order to survey the landscape of existing technologies in order to determine if the existing technologies at that time fulfilled the necessary requirements so that the end product actually could be developed using them.

In order to organize the research within the group, the members split up in a number of areas which connected with the core problems stated within the project. Each area included two collaborating project members. The task was to evaluate and compare the tools for accomplishing the sub-goal of the area. For instance, the data storage area had to investigate the possibilities for storing data locally, in terms of file size, browser availability, and a number of other factors. This included researching APIs and prototyping actual test cases implementing isolated forms of future system features. In the end of the research phase, each area was to present the results and conclusions for the rest of the group. The research areas were divided as follows:

\begin{itemize}
\item Data storage – Investigated how to store data locally for each node. This was crucial in order to meet with the overall requirement of building a decentralized system. 
\item Communication – Reviewed the possibilities for communicating and sending data with peer-to-peer technology between two nodes.
\item Authentication and Permissions – How to solve authentication for nodes.
  \item Prototyping – Rapidly produced a rough test case for sending a file from one node to another.
\end{itemize}
Data storage
There are essentially three alternatives for persisting data on the client: 

LocalStorage
IndexedDB/WebSQL
FileSystem API

LocalStorage is included in the HTML5 WebStorage specification, and is a basic key-value store with a simplistic API. It is supported across all major browsers and has a maximum storage limit of 5 megabytes. The latter was a deal-breaker, since the product would have to support larger files than could possibly fit into that space. LocalStorage further does not support complex structures and indexing, and storing different data types is complex and needs manual serialization and deserialization. Thus this solution was thus rejected at once.

IndexedDB and WebSQL are both client side databases and more sophisticated storage solutions than LocalStorage. WebSQL is supported by Google Chrome, Apple Safari (desktop and iOS), Opera and Android. The specification is not longer maintained by W3C, and will probably be deprecated on all browsers in the future. IndexedDB is supported by all major browsers except for Safari (desktop and iOS), and is a Candidate Recommendation of W3C. Arbitrary types of data can be stored in the database, such as strings, numbers, Javascript objects, and raw binary data. 

The FileSystem API is a collection of methods for reading and writing to a sandboxed filesystem from a browser with client Javascript code. It is a very early standard, and is currently only supported by Google Chrome and Opera, and has the status of Working Draft by W3C. While FileSystem has good performance for larger files and a well-performing asynchronous API, it lacks support for indexing and search. Mozilla seems to have no plans on implementing FileSystem in Firefox.

All of the mentioned technologies are sandboxed: the data is tied to a single origin (http://test.domain.com for instance). All future access to the data must come from that domain (this includes the protocol and port number as well). The browsers also limits the maximum allowed storage size – the quota. The quota is different for each storage mechanism, and the browser typically asks the user with a dialog if they want to let the app exceed the quota. 

The conclusion was to use IndexedDB for persisted resource storage. It was chosen because of its support by Google Chrome, Internet Explorer and Firefox, and due to the fact that its actively maintained (while WebSQL is not). Users of the Safari browser will not be able to utilize the product, but in respect to the project’s overall direction in regards to experimental technologies, this is negligible. 
Communication
Authentication and Permissions
Prototyping
The aim for the prototyping group was to quickly decide if it was in any way possible to achieve the requirements with the technologies chosen. Thus was a rough prototype of Rymd created, which implemented two basic test cases: storing a file in the chosen data storage implementation, and sending that file to another node where it was stored in that node’s local data storage. The prototype worked successfully, which validated the group’s choice of technologies.
j
