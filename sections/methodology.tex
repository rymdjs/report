Initially the project was split into two parts: an evaluation phase and an implementation phase. This chapter intends to further detail the results of each phase and how it relates to the end product.

\section{Evaluation of technologies}

The goal of the evaluation phase was to map out the landscape of relevant technologies. Different options were compared  to each other in order to analyze strengths and weaknesses in regards to a set of given parameters:

\begin{description}
  \item[Suitability]. How well does the technology suit the needs and demands of the job? Are there any technical limitations?
  \item[Maintenance]. Is the technology actively maintained? If not, does it pose an issue? What are the future scenarios?
  \item[Industry support]. Are some unsupported browsers neglible? What are the industry's current opinions?
\end{description}

Research was made about open web technologies, mainly belonging to the HTML5 standard. The usage of open web technologies was a requirement of the project's end product, Rymd, which meant that no native code could be written as part of the system and that the quality of the product would be completely dependent on the state of existing APIs and tools of web development. Thus research was made in order to survey the landscape of existing technologies at the time in order to determine which, if any, fulfilled the requirements so that the end product actually could be developed using them.

A set of areas were created, where each area connected with one or several core problems stated within the project. In each area, evaluation and comparisons were made, which included researching APIs and prototyping actual test cases implementing isolated forms of future system features. The research areas were divided as follows:

\begin{description}
\item[Data storage]. Investigated how to store data locally for each node. This was crucial in order to meet with the overall requirement of building a decentralized system.
\item[Communication]. Reviewed the possibilities for communicating and sending data with peer-to-peer technology between two nodes.
\item[Authentication and Permissions]. How to solve authentication for nodes.
\item[Prototyping]. Rapidly produced a rough test case for sending a file from one node to another.
\end{description}

\subsection{Data storage}

There are essentially four alternatives for persisting data on the client:

\begin{itemize}
\item LocalStorage
\item IndexedDB
\item WebSQL
\item FileSystem API
\end{itemize}

LocalStorage is included in the HTML5 WebStorage specification \cite{WebStorage:Online}, and is a basic key-value store with a simplistic API. It is supported across all major browsers and has a maximum storage limit of 5 megabytes. The latter was a deal-breaker, since the product would have to support larger files than could possibly fit into that space. LocalStorage further does not support complex structures and indexing, and storing different data types is complex and needs manual serialization and deserialization. Thus this solution was rejected at once.

IndexedDB and WebSQL are both client side databases and more sophisticated storage solutions than LocalStorage. WebSQL is supported by Google Chrome, Apple Safari (desktop and iOS), Opera and Android. The specification is not longer maintained by W3C \cite{WebSQL:Online}, and will probably be deprecated on all browsers in the future. IndexedDB is supported by all major browsers except for Safari (desktop and iOS), and is a Candidate Recommendation by W3C \cite{IndexedDB:Online}. Arbitrary types of data can be stored in the database, such as strings, numbers, Javascript objects, and raw binary data.

The FileSystem API is a collection of methods for reading and writing to a sandboxed filesystem from a browser with client Javascript code. It is a very early standard, and is currently only supported by Google Chrome and Opera, and has the status of Working Draft by W3C \cite{FileSystem:Online}. While FileSystem has good performance for larger files and a well-performing asynchronous API, it lacks support for indexing and search. Mozilla seems to have no plans on implementing FileSystem in Firefox \cite{MozillaFileSystem:Online}.

All of the mentioned technologies are sandboxed: the data is tied to a single origin (\emph{http://test.domain.com} for instance). All future access to the data must come from that domain (this includes the protocol and port number as well). The browsers also limits the maximum allowed storage size – the quota. The quota is different for each storage mechanism, and the browser typically asks the user with a dialog if they want to let the app exceed the quota.

\begin{table}
    \begin{tabular}{|l|l|l|l|l|}
    \hline
                       & IndexedDB & WebSQL & File System & LocalStorage \\ \hline
    Google Chrome               & Yes       & Yes    & Yes  & Yes                      \\ \hline
    Mozilla Firefox             & Yes       & \cellcolor{red}No  & \cellcolor{red}No   & Yes                        \\ \hline
    Apple Safari                & \cellcolor{red}No        & Yes  & \cellcolor{red}No    & Yes                        \\ \hline
    Opera                       & Yes       & Yes    & Yes  & Yes                      \\ \hline
    Microsoft Internet Explorer & Yes       & \cellcolor{red}No  & \cellcolor{red}No   & Yes                        \\ \hline
    \end{tabular}
    \caption {Browser support for selected HTML5 APIs}
\end{table}

The conclusion was to use IndexedDB for persisted resource storage. It was chosen because of its support by Google Chrome, Internet Explorer and Firefox, and due to the fact that its actively maintained (while WebSQL is not). Users of the Safari browser will not be able to utilize the product, but in respect to the project's overall direction in regards to experimental technologies, this is negligible.

\subsection{Communication}

% TODO

\subsection{Authentication and Permissions}

% TODO

\subsection{Prototyping}

The aim for the prototyping area was to quickly decide if it was in any way possible to achieve the requirements with the technologies chosen. Thus was a rough prototype of Rymd and Shuttle created, which implemented two basic test cases: storing a file in the chosen data storage implementation, and sending that file to another node where it was stored in that node's local data storage. The prototype worked successfully, which validated the choice of those special technologies.

\section{Implementation}

During the implementation phase, which stretched from the end of the evaluation to the end of the project, the product was implemented according to the requirements. It was early decided that the implementation process would lightly apply agile methodologies. For this project, this included having a Product Backlog with User Stories, working in sprints, and having bi-weekly Scrum-meetings where current state and eventual problems were brought up. For managing the stories in the backlog, the online management tool Pivotal Tracker was used.

All source code was managed by the distributed source versioning system git\footnote{http://git-scm.com/} and with the online tool GitHub\footnote{https://github.com/rymdjs}. It was decided to split up the Rymd library into several smaller code repositories (“modules”), which all resided on GitHub. See section~\ref{sec:system} for details regarding the technical implementation.
