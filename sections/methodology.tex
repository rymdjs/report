Initially the project was split into two parts: an evaluation phase and an implementation phase. This chapter intends to further detail the results of each phase and how it relates to the end product.

\section{Evaluation of technologies}

The goal of the evaluation phase was to map out the landscape of relevant technologies. Different options were compared  to each other in order to analyze strengths and weaknesses in regards to a set of given parameters:

\begin{description}
  \item[Suitability]. How well does the technology suit the needs and demands of the job? Are there any technical limitations?
  \item[Maintenance]. Is the technology actively maintained? If not, does it pose an issue? What are the future scenarios?
  \item[Industry support]. Are some unsupported browsers neglible? What are the industry's current opinions?
\end{description}

Research was made about open web technologies, mainly belonging to the HTML5 standard. The usage of open web technologies was a requirement of the project's end product, Rymd, which meant that no native code could be written as part of the system and that the quality of the product would be completely dependent on the state of existing APIs and tools of web development. Thus research was made in order to survey the landscape of existing technologies at the time in order to determine which, if any, fulfilled the requirements so that the end product actually could be developed using them.

A set of areas were created, where each area connected with one or several core problems stated within the project. In each area, evaluation and comparisons were made, which included researching APIs and prototyping actual test cases implementing isolated forms of future system features. The research areas were divided as follows:

\begin{description}
\item[Data storage]. Investigated how to store data locally for each node. This was crucial in order to meet with the overall requirement of building a decentralized system.
\item[Communication]. Reviewed the possibilities for communicating and sending data with peer-to-peer technology between two nodes.
\item[Authentication and Permissions]. How to solve authentication for nodes.
\item[Prototyping]. Rapidly produced a rough test case for sending a file from one node to another.
\end{description}


\section{Prototyping}

The aim for the prototyping area was to quickly decide if it was in any way possible to achieve the requirements with the technologies chosen. Thus was a rough prototype of Rymd and Shuttle created, which implemented two basic test cases: storing a file in the chosen data storage implementation, and sending that file to another node where it was stored in that node's local data storage. The prototype worked successfully, which validated the choice of those special technologies.

\section{Implementation}

During the implementation phase, which stretched from the end of the evaluation to the end of the project, the product was implemented according to the requirements. It was early decided that the implementation process would lightly apply agile methodologies. For this project, this included having a Product Backlog with User Stories, working in sprints, and having bi-weekly Scrum-meetings where current state and eventual problems were brought up. For managing the stories in the backlog, the online management tool Pivotal Tracker was used.

All source code was managed by the distributed source versioning system git\footnote{http://git-scm.com/} and with the online tool GitHub\footnote{https://github.com/rymdjs}. It was decided to split up the Rymd library into several smaller code repositories (“modules”), which all resided on GitHub. See section~\ref{sec:system} for details regarding the technical implementation.
