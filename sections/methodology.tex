\section{Evaluation of technologies}

The goal of the evaluation phase was to map out the landscape of relevant technologies. Different options were compared with each other in order to analyze strengths and weaknesses in regards to a set of given parameters:

\begin{description}
  \item[Suitability] How well does the technology suit the needs and demands of the job? Are there any technical limitations?
  \item[Maintenance] Is the technology actively maintained? If not, does it pose an issue? What are the future scenarios?
  \item[Industry support] Are some unsupported browsers neglible? What are the industry's current opinions?
\end{description}

Research was done concerning open web technologies, mainly those belonging to the HTML5 standard. The usage of open web technologies was a requirement of the project's end product, Rymd, which meant that no native code could be written as part of the system and that the quality of the product would be completely dependent on the state of existing APIs and tools for web development. Thus research was done in order to survey the landscape of existing technologies in order to determine which, if any, fulfilled the requirements so that the end product actually could be developed using them. The result of this research is presented in chapter \ref{chap:tech_background} with the resulting analysis and discussion in chapter \ref{chap:design}.

A set of areas was created, where each area was connected with one or several core problems stated within the project. In each area, evaluation and comparisons were made, which included researching APIs and prototyping actual test cases implementing isolated forms of future system features. The research areas were divided as follows:

\begin{description}
\item[Data storage] How to store data locally on the client.
\item[Communication] Possibilities for communicating and sending data with peer-to-peer technology between two nodes.
\item[Authentication and Permissions] How to solve authentication between nodes.
\item[Prototyping] The development of a rough test case for sending a file from one node to another.
\end{description}

\section{Prototyping}

The aim for the prototyping phase was to quickly decide if it was in any way possible to achieve the requirements with the technologies chosen. Therefore a rough prototype of Rymd and Shuttle was created, which implemented two basic test cases: storing a file in the chosen data storage implementation and sending that file to another node where it was stored in that node's local data storage. The prototype worked successfully, which validated the choice of  the particular technologies used. The choices that proved successful were carried over to the next step.

\section{Implementation}

At an early stage it was decided that the implementation process would apply light agile methodologies. For this project, this included having a Product Backlog with User Stories, working in sprints and having semi-weekly Scrum-meetings where current state and eventual problems were brought up.

All source code was managed by the distributed source versioning system git\footnote{http://git-scm.com/} and hosted at the online service GitHub\footnote{https://github.com/rymdjs}.

\subsection{Modularity}
\label{sec:modularity}

As stated in section~\ref{sec:purpose}, developers should be able to easily incorporate their own preferred implementations of the system's core functionality. For modularity to be properly fulfilled, features that could have alternative implementations had to be clearly identified and separated into individual code repositories referred to as \emph{modules}.

Accomplishing this would allow developers to not only supply more fitting modules to their own end products but also easily exchange existing ones if better alternatives were to be released. This has been particularly important in Rymd since it utilizes technologies at the web's furthest frontier; unfinished drafts in constant change.
