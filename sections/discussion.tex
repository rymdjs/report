
\section{Could a truly decentrilized system be achieved?}
One of the main initial goals with Rymd was to make the system truly decentralized and independent of any web servers. The Shuttle application can be downloaded and run locally, and all data transfers between clients are performed in a peer-to-peer fashion. There is, however, two parts where communicatin with a central endpoint needs to be done:
\subsection{WebRTC ICE}
As described in section~\ref{sec:p2p}, establishing of WebRTC connections still relies on the availability of a STUN or TURN server. This makes implementing applications depend on the availability of such a server. However, there are several public ICE servers available and in the case of downtime it is trivial to set up a new one and make the application use the new server instead. Also, since verification of identities of peers is performed locally and all data is end-to-end encrypted, there is no possibility of the administrator of these servers to spoof identities or deduce anything about shared resources. The two things that do leak are identity names (since these are needed to deduce who to connect to whom) and, if TURN is used, estimated size of data transferred. We found no way around the former and found it to be a fair tradeoff - future implementations could 

\subsection{Accessing the DHT}
Since Namecoin (or any other currently existing cryptocurrency for that matter) communicates using their own binary data protocol\cite{BitcoinSource:2014:Online}, Rymd can not interact directly with the blockchain and fetch public keys for identities. Therefore, an HTTP gateway running the Namecoin software was developed that acts as a bridge between the blockchain and Rymd nodes. Trust in the operator of this gateway is crucial, since public keys are fetched and verified through it. The paranoid user could easily run their own gateway, or manually verify or insert public keys using their own Namecoin client. On May 6th, at the time of writing of this report, a public and more general HTTP/Blockchain interface at \emph{chain-api.com}\cite{Chain:2014:Online} was released. Currently it only support Bitcoin, but promises future integration with the Namecoin blockchain. Once that happens, it would be trivial to replace the current gateway with Chain if one would like to do so. The buzz around services like these shows that this is an emerging area.

Something that would both solve these issues and be very interesting in many other areas is a blockchain where nodes can communicate through open web protocols. This would mean that web clients could interact directly with the blockchain without interacting through gateways like these, at the same time allowing them to contribute to the network. Given the premises stated in the introduction and the rapid development of emerging blockchains and cryptocurrencies, we think it is only a matter of time before this happens.

\section{Is all data truly cryptographically secured?}
%TODO: Reference
In the data transport layer, all communicaion over WebRTC is DTLS encrypted. As stated in section XX, for local storage and sending of resources, every resource is also AES encrypted with a resource-specific key. The only web browser that has a working implementation of the Web Crypto API is Google Chrome. Since the implementation is still experimental and there is not yet support for secure storage of cryptographic keys, the keys are stored alongside their encrypted data in IndexedDB. As long as the keys are not passphrase protected, this effectively means that at the level of the local client, the encryption adds no extra protection and can be considered reduntant. An adversary gaining access to the database with the encrypted data would also have access to the decryption key. This is most likely only during a transitional period and as the Web Crypto API becomes stable and fully implemented across browsers, separate key storage options will become available and this can be resolved.

Despite this problem, the AES encryption still serves a purpose. Consider an application utilizing Rymd where users communicate through each other in a "darknet" fashion. In these cases it is imperative that resources can be transmitted separately from their keys and metadata so that intermediate peers can facilitate the transfer of resource data without gaining knowledge of the contents.

Also, systems with updateable resources can and should regenerate keys for each version and backward secrecy - the property that access to the key for one version of the resource will not allow decryption of older versions of that resource - will be achieved.


\section{}

\section{Quality of the resulting product}


