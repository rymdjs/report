% TODO:
%   - Self-critical somewhere
%   - Ethical insights

\section{Could a truly decentralized system be achieved?}
One of the main initial goals with Rymd was to make the system truly decentralized and reliable independently of the availability of certain services. To a large extent, Rymd is successful in this area. Any application, including the prototype Shuttle, can be downloaded and executed locally. Therefore there are no dependencies on web servers, since all data transfers between clients are performed in a peer-to-peer fashion. No central database outside of the local client stores persistent data. There is, however, two parts where communication with central endpoints still needs to be done:

\subsection{WebRTC ICE}
As described in section~\ref{sec:p2p}, establishing of WebRTC connections still relies on the availability of a STUN or TURN server. This makes implementing applications depend on the availability of such a server. However, there are several public ICE servers available and in the case of downtime it is trivial to set up a new one and make the application use the new server instead. Also, since verification of identities of peers is performed locally and all data is end-to-end encrypted, there is no possibility of the administrator of these servers to spoof identities or deduce anything about shared resources. The two things that do leak are identity names (since these are needed to deduce who to connect to whom) and, if TURN is used, estimated size of data transferred. We found no way around the former and decided the latter to be a fair tradeoff - future implementations that care about leaking of resource size could solve this by also transfer redundant padding data regardless of resource size. 

\subsection{Accessing the DHT}
The Namecoin blockchain is used to tie identities to their public keys and PeerJS endpoints. Since Namecoin (or any other currently existing cryptocurrency for that matter) communicates using their own binary data protocol\cite{BitcoinSource:2014:Online}, Rymd can not interact directly with the blockchain to fetch this information before a mutual peer-to-peer authenticated connectin is established. Therefore, an HTTP gateway running the Namecoin software was developed that acts as a bridge between the blockchain and Rymd nodes. Trust in the operator of this gateway is crucial, since public keys are fetched and verified through it. The paranoid user could, however, easily run their own gateway or manually verify or insert public keys using their own Namecoin client. On May 6th, at the time of writing of this report, a public and more general HTTP/Blockchain interface at \emph{chain-api.com}\cite{Chain:2014:Online} was released. Currently it only support Bitcoin, but promises future integration with the Namecoin blockchain. Once that happens, it would be trivial to replace the current gateway with Chain if one would like to do so. The buzz around services like these shows that this is an emerging area with more interesting development to come in the near future.

Something that would both solve these issues and be very interesting in many other areas is a blockchain where nodes can communicate through open web protocols. This would mean that web clients could interact directly with the blockchain without going through external gateways like these, at the same time allowing them to contribute to the network. Given the premises stated in the introduction and the rapid development of emerging blockchains and cryptocurrencies, we think it is only a matter of time before this happens.

\section{Is all data truly cryptographically secured?}
%TODO: Reference
In the application layer, all communication over WebRTC is DTLS encrypted. As stated in section XX, for local storage and sending of resources, every resource is also AES encrypted with a resource-specific key. Since all existing browser implementations of the Web Cryptography API are still experimental and partial and there is not yet support for secure storage of cryptographic keys, the keys are stored alongside their encrypted data in IndexedDB. As long as the keys are not passphrase protected, this effectively means that at the level of the local client, the encryption adds no extra protection and can be considered redundant. An adversary gaining access to the database with the encrypted data would also have access to the decryption key. This is most likely only during a transitional period and as the Web Crypto API becomes stable and fully implemented across browsers, separate key storage options will become available and this can be resolved.

Despite this problem, the AES encryption still serves a purpose. Consider an application utilizing Rymd where users communicate through each other in a \emph{darknet} fashion. In these cases it is imperative that resources can be transmitted separately from their keys and metadata so that intermediate peers can facilitate the transfer of resource data without gaining knowledge of the contents.

Also, systems with updateable resources can and should regenerate keys for each version and backward secrecy - the property that access to the key for one version of the resource will not allow decryption of older versions of that resource - will be achieved.

\section{Is the system modular and implementation agnostic?}
As much as Rymd uses and relies some of the latest web technologies, great care was taken during development to not make it rely on any of these implementations, should they be superceded or complemented by other, more fitting alternativces. The main Rymd library itself handles only the business logic of the system and gets the modules implementing data storage, cryptography, peer-to-peer communication and DHT interaction supplied at runtime via dependency injection. Developers who have their own idea of how these needs should be served in their projects could write their own implementation modules. The one area where work needs to be done is that currently, storage of keys, metadata and resource data are tied together. Before Rymd can go stable, this should be addressed by treating these as separate data stores alltogether even if the current implementation puts all three side by side in IndexedDB.

\section{Ethical aspects}
Rymd was originally conceived from an ethical issue: The one that free, private and secret communication should be easily accessible and usable on the web. As it currently stands, truly secret and private communication requires running binary files and/or putting trust in a service provider. Rymd aims to be a step away from that restriction.

\subsection{Does Rymd meet this goal?}
At its current state, Rymd should not be trusted with confidential data. This is mainly because of the limitations stated in the preceding sections that come from the choice of still immature, cutting-edge technnologies. Also, Rymd is still in an experimental stage and should not be considered stable or trusted until it has been exposed to extensive peer review and scrutiny by the community - a reservation that should be held for any project of this nature. However, we are confident that we are going in the right direction and hope that further development could make Rymd to a contributor in the movement of free communication on the web.

\subsection{Implications}
As always when it comes to services enabling private communication, concerns are raised on the issue of what they can be used for. Commonly mentioned are terrorism, drug dealing and child pornography. First, we want to emphasize that Rymd does not in itself provide any anonymity for its users (though it could easily be used in conjunction with anonymization services such as TOR). While Rymd could indeed be used for these purposes, there are already other services such as the ones mentioned in~\ref{sec:similar} that are currenly used for these purposes - Rymd does not enable risks that are not already present. Mainly, however, it is our opinion that the right to private and secure communication without corporation or government surveillance is a human right, and this right is effectively nonexistant if it requires significant monetary resources and/or technical know-how. Putting this standpoint aside, we have mainly treated this issue as a technical one and will let the reader of this report decide for themselves where they stand and how Rymd relates to this.

