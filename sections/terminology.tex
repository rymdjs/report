\section*{General terminology and abbreviations}

\begin{description}
  \item[Adobe PhoneGap] A software enabling development of cross-platform hybrid smartphone applications - applications developed using web technologies, but packaged as native smartphone binaries.
  \item[API] Application Programming Interface. An interface that software developers can use to get easy access to data and/or particular functionality for their software. Typically exposed as a software library or HTTP service.
  \item[Bitcoin] The first and biggest widespread cryptocurrency.
  \item[CA] Certificate Authority. A third party that is trusted to verify the validity of public keys and certificates.
  \item[Chrome Apps] Similar to Adobe PhoneGap, but for desktop applications running in a Google Chrome sandbox.
  \item[Cloud storage] A service that hosts data externally with seamless access over the internet.
   (called \emph{blockchain}) to track transactions. The vast majority of cryptocurrencies are forks off Bitcoin.
  \item[Centralized system] A system which has several nodes connecting to and depending on one or a few central endpoints.
  \item[CRUD] Create-Read-Update-Delete. A set of actions to be taken on data collections.
  \item[Cryptocurrency] A network transaction system that uses a fully distributed cryptographically secured ledger
  \item[Decentralized system] A system where responsibilities are shared across the nodes and does not depend on a single, central endpoint.
  \item[DHT] Distributed Hash Table. A notion in computer science of a distributed key-value store.
  \item[Firefox OS] A smartphone operating system where all applications are web-based.
  \item[GUID] Globally Unique Identifier. Used as pseudo-unique identifiers, such as keys in a database. Usually 128-bit values stored as 32 hexadecimal in groups separated by hyphens.
  \item[IETF] Internet Engineering Task Force. An organization with the purpose of improving the internet by creating standards. 
  \item[JSON] JavaScript Object Notation, a lightweight alternative to XML for exchanging data (often over different APIs). JSON has become a common way for formatting data, and most languages have native implementations for parsing and serializing JSON.
  \item[NoSQL] All database systems which are not modelled in tabular relations. Examples are graphs, trees, and key-value stores.
  \item[OpenPGP] A standard for data encryption and signing, originally coming from the proprietary software Pretty Good Privacy (PGP) and widely spread through the free implementation GPG.
  \item[P2P] Peer-to-peer. Distributed, direct communication between clients.
  \item[PKCS] A group of standards for public-key cryptography devised and published by RSA Security Inc.
  \item[PKCS\#8] Private-Key Information Syntax Standard, used to carry private certificate keypairs and provide a way to construct private key certificates in ASN1.
  \item[PKI] Public Key Infrastructure. A system that associates (unique) user identities with their public keys. Typically implemented as a Web of Trust or with one or several CAs. Typically, trusted parties use their private keys to sign the public keys of users to verify the connection between an identity and a public key.
  \item[REST] Representational State Transfer, a style used for structuring data APIs by putting constraints on the different URL endpoints. If an API uses REST style, it is often referred to as \emph{RESTful}.
  \item[RSA] A widely used public-key cryptosystem. As such, it builds on pairs of private and public keys where encryption with one can be reversed by decrypting with the other. This system is asymmetric and the security relies on the practical difficulty of factoring the product of two large prime numbers.
  \item[RDBMS] Relational Database Management System, a popular type of database management system based on the relational model.
  \item[SPKI] Simple public key infrastructure, a successor to X.509. It was designed with the goal to eliminate overcomplication and scalability problems.
  \item[SQL] Structured Query Language. A language for managing, querying and manipulating data in relational database systems.
  \item[SQL injection] A technique for injecting malicious SQL code into the executing database queries in order to, for instance, dump the contents of the database.
  \item[XHR] XMLHttpRequest is a web browser JavaScript API used for sending asynchronous HTTP requests directly from the client.
  \item[XSS] Cross Site Scripting is the technique for injecting client-side scripts into web pages.
  \item[X.509] an ITU-T standard for a public key infrastructure (PKI).
  \item[Web of Trust] A type of distributed PKI that builds on peer-to-peer trust. The idea is that if Alice trusts Bob, then Bob is trusted introduce new identities and public keys for Alice.
  \item[W3C] World Wide Web Consortium. A standards organization for the World Wide Web.

\end{description}

\section*{Terms with specific meaning in the Rymd project}
\begin{description}
  \item[Identity] A unique, memorable string identifying a user within the network.
  \item[Module] A delimited area of interest and functionality in system architecture.
  \item[Node] A client in the network (such as a web browser).
  \item[Resource] A file or folder in the network (a thing that can be shared between nodes)
  \item[Rymd] The developer library for web based peer-to-peer sharing – the main product. Is also the Swedish word for \emph{space}, which encompass the main ideas of the project.
  \item[Shuttle] A web based file sharing application implemented using Rymd to demonstrate the basic capabilities of the project.
  %\item[]
\end{description}
