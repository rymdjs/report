\section{IndexedDB}  
The IndexedDB is a transactional, indexed client database capable of storing different types of data structures with an asynchronous API. The specification is a Candidate Recommendation by the W3C, as of July 2013:

\begin{quote}
This document defines APIs for a database of records holding simple values and hierarchical objects. Each record consists of a key and some value. Moreover, the database maintains indexes over records it stores. An application developer directly uses an API to locate records either by their key or by using an index. A query language can be layered on this API. An indexed database can be implemented using a persistent B-tree data structure.
– Indexed Database API, W3C
\end{quote}

IndexedDB is actively developed, and implemented in the latest versions of Mozilla Firefox, Google Chrome, Microsoft Internet Explorer, and Opera. An IndexedDB database includes a set of \emph{object stores}, which act like tables in relational database management systems. An object store can hold \emph{records} of different types, including binary data and Javascript primitives and objects. Each record has a \emph{key} (either specified by the developer or automatically managed by the database) which is used for indexing and retrieving records.

Its API is asynchronous and includes some degree of complexity. Unlike its competitor, WebSQL, IndexedDB does not support the SQL language, and exposes instead ways for querying and manipulating data via API methods and transactions objects. A positive facet of the rejection of SQL in favor of API methods is the prevention of SQL injection attacks, but with the cost of a steeper learning curve for already experienced database developers.

\subsection{Privacy and reliability of IndexedDB}


