\section{WebRTC}
WebRTC is a project which enables different types of peer-to-peer communication in the browser through three APIs: MediaStream, RTCPeerConnection and RTCDataChannel.

\subsection{RTCDataChannel}
The API which demonstrated the capabilitees that Rymd desired was the RTCDataChannel API. This API allows for arbitrary data being sent between peers. Data transfers are secured with the DTLS (Datagram Transport Layer Security) protocol. The DTLS protocol is based on the TLS (Transport Layer Security) protocol, the main difference being that DTLS is constructed for datagrams while TLS is used for more reliable transport protocols like TCP.

Before a connection can be initiated between peers, one of two parts must extend an offer which contains data describing the connection to the other part - this is often referred to as the signaling phase. The signaling phase requires a channel where the offer can be negotiated - the channel is often a dedicated signaling server but examples of a more serverless approach can be found\cite{webrtcsignalserver}. The standard does not provide any recommendations regarding the choice of signaling channel and protocol - this is for developers to decide.

The connection phase is handled by ICE (Interactive Connectivity Establishment).

\subsection{ICE}
Mention how WebRTC can handle the whole ICE workflow, STUN, TURN etc.

\section{IndexedDB}
The IndexedDB is a transactional, indexed client database capable of storing different types of data structures with an asynchronous API. The specification is a Candidate Recommendation by the W3C, as of July 2013:

\begin{quote}
This document defines APIs for a database of records holding simple values and hierarchical objects. Each record consists of a key and some value. Moreover, the database maintains indexes over records it stores. An application developer directly uses an API to locate records either by their key or by using an index. A query language can be layered on this API. An indexed database can be implemented using a persistent B-tree data structure.
– Indexed Database API, W3C
\end{quote}

IndexedDB is actively developed and implemented in the latest versions of Mozilla Firefox, Google Chrome, Microsoft Internet Explorer and Opera. An IndexedDB database includes a set of \emph{object stores}, which act like tables in relational database management systems. An object store can hold \emph{records} of different types, including binary data and Javascript primitives and objects. Each record has a \emph{key} (either specified by the developer or automatically managed by the database) that is used for indexing and retrieving records.

Its API is asynchronous and includes some degree of complexity. Unlike its competitor, WebSQL, IndexedDB does not support the SQL language, and instead exposes ways for querying and manipulating data via API methods and transactions objects. A positive facet of the rejection of SQL in favor of API methods is the prevention of SQL injection attacks, but with the cost of a steeper learning curve for already experienced database developers.

\subsection{Privacy and reliability of IndexedDB}

\section{Namecoin}
A phenomenon that has been on the rise during recent years is that of cryptocurrencies such as Bitcoin \cite{CryptoCoinInsider:2014:Online}. One cryptocurrency that borrows almost everything from Bitcoin is Namecoin\cite{CryptoCoinInsider:2014:Online}. Namecoin is essentially a fork of Bitcoin with new transaction types that allows its blockchain to be utilized as a distributed key-value store. Its main purpose is to be used as a decentralized domain name system, or DNS.

A normal DNS query, e.g. from a browser wanting to acquire the IP address corresponding to the domain name of a web server, ultimately relies on lookups in servers managed by ISPs or other major central authorities \cite{CryptoCoinInsider:2014:Online}. Domain name distribution is managed by third party companies known as registrars and these in turn are governed by the central authorities responsible for specific top level domains, or TLDs. The TLD is the last part of an ip-adress such as ".com" or ".se".

With a decentralized DNS such as Namecoin, top level domains can exist without being controlled by anyone\cite{CryptoCoinInsider:2014:Online}. Also, the DNS lookup tables where domain names and their IP addresses are stored are shared in a peer-to-peer manner. The only necessary condition for these domains to be accessible is that there are participants willing to run the DNS server software for everybody else. All in all, this peer-to-peer top level domain is impossible to control by central authorities by means other than obtaining the actual servers running the software. This of course could be both used and abused since authorities’ dirty secrets could be made visible for everyone to see, but so could malicious content such as child abuse and terrorism.
