% TODO May need rework.
Standards and technologies for web applications are being rapidly developed, and the boundaries for what is possible to achieve in a web application is being continously pushed by browser vendors and standards groups. Web based applications are today a viable alternative, unless the application is for mobile and is in need for heavy graphics performance, like for photo/video editing and games, or basically everywhere where sophisticated memory management is required. In \emph{Why Mobile Web Apps Are Slow} \cite{MobileApps:Online} it is argumented how Javascript is not yet suitable for heavy performing applications, but this is not an issue for this project, as it is not included in the product's nature or scope at all.

Of relevance to Rymd, the Web Real-Time Communications (WebRTC) protocol \cite{WebRTC:Online} has become usable for arbitrary data streams in major browsers in 2014. Also of relevance, the still unfinalized Web Crypto API \cite{WebCrypto:Online} is currently available in an experimental stage in recent months at the time of this writing. The goals of Rymd has thus become technically viable in a web environment as of very recently.

The development of client side data storage in HTML5 is also an area that has become more stable and supported across browsers and vendors. Web applications can utilize offline storage like databases (WebSQL and IndexedDB), key value stores (LocalStorage), and even offline caching with Application Cache \cite{OfflineWebApps:Online} in order to build completely offline applications with no requirements for internet access.

Furthermore, there is currently a lot of work underway in the field of distributed secret communication. Notable projects that share similar ideas or have inspired Rymd are mentioned under section~\ref{sec:similar}. Also worth noting is the new field of cryptocurrencies, which work by distributing a cryptographically based ledger over an entire network. The first and most well-known is Bitcoin \cite{Bitcoin:Online}, but there are also subsequent currencies that extend the original idea beyond that of a traditional currency to a system that can be used for a wide range of applications. The first worth noting is Namecoin \cite{Namecoin:Online}, which adds a global and secure key-value store. Another interesting initiative is Ethereum \cite{Ethereum:Online}, a cryptocurrency with contracts that not only allow storage of arbitrary data in the blockchain, but can also be scripted with a Turing complete programming language and can therefore be used to implement arbitrary systems. A system like Ethereum could be very interesting to explore for a project like Rymd, but it is still in such an early stage that it is deemed too unstable to be useful at this point. Namecoin is currently considered a good candidate for key distribution. Keybase \cite{Keybase:Online} is another recent initiative that intends to solve the distribution of public keys. It is essentially a HTTP-interface that maps keys to identities. While commandable, it again raises the issue of centralized storage. Systems relying on Keybase put a lot of trust on the availability and integrity of their service.

\section{Client-side storage}

% Mention WebSQL, IndexedDB, etc.
% TODO Johan

\section{Distributed storage}

% Mention Namecoin, others?
% TODO Robert

\section{Peer-to-Peer networking}

% Mention WebSockets, others?
% TODO Niklas & Robin

\section{Cryptography}

% Talk about theoretical cryptography, certificates, etc.
% TODO Johannes & Robert
